This element allows simulation of a transverse impedance using a
``broad-band'' resonator or an impedance function specified in a file.
The impedance is defined as the Fourier transform of the wake function
\begin{equation}
Z(\omega) = \int_{-\infty}^{+\infty} e^{-i \omega t} W(t) dt
\end{equation}
where $i = \sqrt{-1}$, $W(t)=0$ for $t<0$, and $W(t)$ has units of
$V/C/m$.  Note that there is no factor of $i$ in front of the
integral.  Thus, in {\tt elegant} the transverse impedance is simply
the Fourier transform of the wake.  This makes it easy to convert data
from a program like ABCI into the wake formalism using {\tt sddsfft}.

For a resonator impedance, the functional form is
\begin{equation}
Z(\omega) = \frac{-i\omega_r}{\omega} \frac{R_s}{1 + iQ(\frac{\omega}{\omega_r} - \frac{\omega_r}{\omega})},
\end{equation}
where $R_s$ is the shunt impedance in $Ohms/m$, $Q$ is the quality
factor, and $\omega_r$ is the resonant frequency.

When providing an impedance in a file, the user must be careful to conform to these
conventions.
In addition, the units of the frequency column must be Hz, while the units
of the impedance components must be Ohms/m.
At present, {\tt elegant} does not check the units for correctness.

Other notes:
\begin{enumerate}
\item The frequency data required from the input file is {\em not} $\omega$, but rather
  $f = \omega/(2 \pi)$.
\item The default smoothing setting ({\tt SG\_HALFWIDTH=4}), may apply too much smoothing.
  It is recommended that the user vary this parameter if smoothing is employed.
\item Impedance data can be created from a wake function using the script \verb|trwake2impedance|,
  which is supplied with \verb|elegant|. This script also illustrates how to scale the data 
  with the frequency spacing. The script uses \verb|sddsfft|, which produces a
  folded FFT ($f\ge 0$) from a real function. The folded FFT representation involves multiplying the
  non-DC terms by 2. \verb|elegant| expects this and internally multiplies the DC term by 2 as well.
\item Using the broad-brand resonator model can often result in a very large number of bins
 being used, as {\tt elegant} will try to resolve the resonance peak and achieve the desired
 bin spacing. This can result in poor performance, particularly for the parallel version.
\item Wake output is available only in the serial version.
\end{enumerate}

Bunched-mode application of the impedance is possible using specially-prepared input
beams. 
See Section \ref{sect:bunchedBeams} for details.
The use of bunched mode for any particular \verb|ZTRANSVERSE| element is controlled using the \verb|BUNCHED_BEAM_MODE| parameter.
