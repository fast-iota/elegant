For a discussion of the method behind this element, see M. Borland,
``Simple method for particle tracking with coherent synchrotron
radiation,'' Phys. Rev. ST Accel. Beams 4, 070701 (2001) and
G. Stupakov and P. Emma, SLAC LCLS-TN-01-12 (2001).

{\bf Recommendations for using this element.}  The default values for
this element are not the best ones to use.  They are retained only for
consistency through upgrades.  In using this element, it is
recommended to have 50 to 100 k particle in the simulation.  Setting
{\tt BINS=600} and {\tt SG\_HALFWIDTH=1} is also recommended to allow
resolution of fine structure in the beam and to avoid excessive
smoothing.  It is strongly suggested that the user vary these
parameters and view the histogram output to verify that the
longitudinal distribution is well represented by the histograms (use
{\tt OUTPUT\_FILE} to obtain the histograms).  For LCLS simulations,
we find that the above parameters give essentially the same results as
obtained with 500 k particles and up to 3000 bins.

In order to verify that the 1D approximation is valid, the user should
also set {\tt DERBENEV\_CRITERION\_MODE = ``evaluate''} and view
the data in {\tt OUTPUT\_FILE}.  Generally, the criterion should be
much less than 1.  See equation 11 of \cite{Derbenev}.

In order respects, this element is just like the {\tt CSBEND} element,
which provides a symplectic bending magnet that is accurate to all
orders in momentum offset. Please see the manual page for {\tt CSBEND}
for more details about features not related to CSR.

{\bf Splitting dipoles}: 
Splitting dipoles with continuation of CSR effects is possible provided the dipole sections (all of which must have the
same name) are either consecutive or separated only by \verb|MARK|, \verb|WATCH|, or \verb|LSCDRIFT| elements.
The \verb|LSCDRIFT| elements must have \verb|L=0| and should have \verb|LEFFECTIVE| set to the length of the upstream
dipole segment.
This allows simulating LSC and CSR within a single dipole.



