The \verb|STICKY| parameter results in the aperture contour being applied inside subsequent \verb|CCBEND|,
\verb|CSBEND|, \verb|CSRCSBEND|, \verb|KQUAD|, \verb|KSEXT|, \verb|KOCT|, and \verb|KQUSE| elements, as well as at the
end of other downstream elements. This continues until another \verb|APCONTOUR| element asserts a new contour, or uses
\verb|CANCEL=1| to cancel the feature.

For versions 2022.1 and later, the input file may have multiple pages, each with $(x, y)$ points specifying a
closed contour. The combination of the effect of these contours is specified using the \verb|Logic| parameter in
the input file, utilizing the logical stack and logical operators of the \verb|rpn| module.
For each particle, the program loops over each contour.
If the particle is inside (outside) the contour, a \verb|true| (\verb|false|) value is pushed onto the stack,
followed by execution of the indicated logic.
If the final value is \verb|true|, the particle survives,  otherwise it is lost.
(The \verb|invert| parameter on the element definition can be used to invert this.)

For example, if the aperture file is
\begin{verbatim}
SDDS1
&column name=x type=float units=m &end
&column name=y type=float units=m &end
&parameter name=Logic type=string &end
&data mode=ascii no_row_counts=1 &end
""
-0.01   -0.015
-0.01   -0.0075
-0.015   -0.0075
-0.015   -0.015
-0.01   -0.015

"||"
0.01   -0.015
0.01   -0.0075
0.015   -0.0075
0.015   -0.015
0.01   -0.015
\end{verbatim}
then particles survive only if inside one of the two rectangles.
