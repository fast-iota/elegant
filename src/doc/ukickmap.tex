This element provides simulation of undulators using kick maps \cite{Elleaume1992}.
For general-purpose kickmaps, use the \verb|KICKMAP| element.

A script (\verb|km2sdds|) is provided with the {\tt elegant}
distribution to translate RADIA \cite{radia} output into SDDS for use by
\verb|elegant|.

The input file has the following columns:
\begin{itemize}
\item \verb|x| --- Horizontal position in meters.
\item \verb|y| --- Vertical position in meters.
\item \verb|xpFactor| --- Horizontal kick factor $C_x$ in $T^2 m^2$.  This factor is defined by
equation (5a) in \cite{Elleaume1992}.  In particular, $\Delta x^\prime = C_x/H^2$, where
$H$ is the beam rigidity in  $T^2 m^2$.
\item \verb|ypFactor| --- Vertical kick factor $C_y$ in $T^2 m^2$. This factor is defined by
equation (5b) in \cite{Elleaume1992}.  In particular, $\Delta y^\prime = C_y/H^2$, where
$H$ is the beam rigidity in  $T^2 m^2$.
\end{itemize}
The values of \verb|x| and \verb|y| must be laid out on a grid of equispaced points.
It is assumed that the data is ordered such that \verb|x| varies fastest.  This can be
accomplished with the command
\begin{verbatim}
% sddssort -column=y,increasing -column=x,increasing input1.sdds input2.sdds
\end{verbatim}
where \verb|input1.sdds| is the original (unordered) file and \verb|input2.sdds| is the
new file, which would be used with \verb|UKICKMAP|.

The data file is assumed to result from integration through a full device. 
If instead it results from integration through just a single period of a full device, one
should set the \verb|SINGLE_PERIOD_MAP| parameter to 1 and \verb|N_KICKS| equal to the
number of periods.  (One can also use the \verb|FIELD_FACTOR| parameter to get the same
result, but this is confusing and is discouraged.)

{\tt elegant} performs radiation integral computations
for \verb|UKICKMAP| and can also include radiation effects in
tracking.  This feature has limitations, namely, that the radiation
integral computations assume the device is horizontally deflecting.
However, in tracking, no such assumption is made.  
To obtain synchrotron radiation integral effects (e.g., in output from \verb|twiss_output|), 
the \verb|KREF| and \verb|PERIODS| parameters must be given.
Care must be taken when using the \verb|FIELD_FACTOR| parameter in this case, particularly if
it is adjusted to account for using a single-period kickmap multiple times.
To obtain synchrotron radiation effects in tracking, the \verb|SYNCH_RAD| and/or \verb|ISR| flags
must additionally be used.

N.B.: at present this element is {\em not} included in beam moments
computations via the \verb|moments_output| command (the \verb|CWIGGLER| element
is an option for that).

The \verb|YAW| and \verb|YAW_END| parameters can be used in the simulation of canted IDs.
Normally, steering magnets are used to create an angle between the devices.
The devices are thus oriented in the reference coordinate system, meaning the beam tranverses
the IDs at an angle.
If it is desirable to align the IDs to the beam, the IDs can be yawed. A positive yaw will
tilt the ID so that it is colinear with a beam that has been kicked by a positive horizontal
steering angle.
The \verb|YAW_END| parameter defines which end of the ID is held fixed when the yaw is applied.

This element was requested by W. Guo (BNL), who also assisted with the
implementation and debugging.

