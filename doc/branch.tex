This element is experimental and should be used with care.
It may not work well with other features, e.g., orbit correction or twiss parameter output.
It should work well with tracking.

Use of the \verb|BRANCH| element to change the starting point in the lattice is not ideal.
It is better to use the \verb|change_start| command.

The element permits switching tracking between two segments of a beamline. This can be done once per run
or periodically. For the former, the \verb|COUNTER| parameter should be used to specify the pass number (which
is zero on the first pass) on which to branch.
For the latter, the \verb|INTERVAL| $i$ and (optionally) \verb|OFFSET| $o$ parameters should be used; the branch will occur
when $(p-o) \% i == 0$.

The application that inspired creation of this element is to switch from tracking using lumped elements to tracking
using element-by-element methods.
More specifically, imagine we want to track for 10,000 turns to reach an equilibrium, then perform a beam abort.
The equilibrium state can be accurately and rapidly modeled using lumped elements, such as \verb|ILMATRIX| and \verb|SREFFECTS|,
but the beam abort needs to be modeled using comparatively slow element-by-element tracking.
\begin{verbatim}
RING1: ILMATRIX,...
SR1: SREFFECTS,...
...
RINGFULL: line=(SECTOR1, SECTOR2, ..., SECTOR40)
M1: MARK
M2: MARK
RF: RFCA,...
BR1: BRANCH,COUNTER=10000,BRANCH_TO="M1"
BR2: BRANCH,COUNTER=-1,BRANCH_TO="M2"
BL: line=(BR1,RING1,SR1,M1,RINGFULL,M2,RF)
\end{verbatim}

Another application is to model a periodic bypass, e.g.,
\begin{verbatim}
RINGA: line=(...)
RINGB: line=(...)
RINGC: line=(...)
BYPASS: line=(...)
M1: MARK
M2: MARK
BR1: BRANCH,INTERVAL=100,BRANCH_TO="M1",ELSE_TO="M2"
BR2: BRANCH,COUNTER=-1,BRANCH_TO="M3"
BL: line=(RINGA,BR1,M1,BYPASS,BR2,M2,RINGB,M3,RINGC)
\end{verbatim}
In this example, the full ring is composed of three sections, \verb|RINGA|, \verb|RINGB|, and \verb|RINGC|.
Every 100 passes, the \verb|RINGB| portion is bypassed in favor of \verb|BYPASS|.

