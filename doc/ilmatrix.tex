This element allows fast, symplectic tracking of transport through a
periodic cell with chromatic and amplitude-dependent tunes, beta
functions, and dispersion.  This is done by computing a linear matrix
for every particle using Twiss
parameters, tunes, dispersion, etc., supplied by the user.  The user
can also supply selected chromatic and amplitude derivatives of these
quantities, which are used to compute the individual particle's beta
functions, tune, dispersion, etc., which in turn allows computing the
individual particle's linear matrix.

The starting point is the well-known expression for the one-turn linear matrix
in terms of the lattice functions
\begin{equation}
R_q = 
\left(\begin{array}{cccc}
\cos 2\pi\nu_q + \alpha_q \sin 2\pi\nu_q & \beta_q \sin 2\pi\nu_q \\
-\gamma_q \sin 2\pi\nu_q                 & \cos 2\pi\nu_q - \alpha_q \sin 2\pi\nu_q     \\
\end{array}\right)
\end{equation}
where $\nu_q$ is the tune in the $q$ plane. We can expand the quantities in the matrix using
\begin{equation}
\nu_q = \nu_{q,0} + \sum_{n=1}^3 \left(\frac{\partial^n \nu_q}{\partial\delta^n}\right)_0\frac{\delta^n}{n!} + 
\sum_{n=1}^2 \left(\frac{\partial^n\nu_q}{\partial A_x^n}\right)_0 \frac{A_x^n}{n!} +
\sum_{n=1}^2 \left(\frac{\partial^n\nu_q}{\partial A_y^n}\right)_0 \frac{A_y^n}{n!} +
\left(\frac{\partial^2 \nu_q}{\partial A_x\partial A_y}\right)_0 A_x A_y
\end{equation}
where $\delta = (p-p_0)/p_0$ is the fractional momentum offset,
$A_q = (q_\beta^2 + (\alpha_q q_\beta + \beta_q q^\prime_\beta)^2)/\beta_q$ is
the betatron amplitude, and the betatron coordinates are computed using
\begin{equation}
q_\beta = q - \delta\left(\eta_q + \left(\frac{\partial \eta_q}{\partial \delta}\right)_0 \delta\right)
\end{equation}
and
\begin{equation}
q^\prime_\beta = q^\prime - \delta \left(\eta^\prime_q + \left(\frac{\partial \eta^\prime_q}{\partial \delta}\right)_0 \delta \right)
\end{equation}
At each turn, $\delta$, $A_x$, and $A_y$ are computed for each particle.
The user-supplied values of the various derivatives are then used to 
compute the tunes for each particle.
Similar expansions are used to compute the other lattice functions.
This allows computing the 2x2 transfer matrices for the betatron coordinates in the 
x and planes, then advancing the
betatron coordinates one turn, after which the full coordinates are recomputed by adding back the
momentum-dependent closed orbit.

The pathlength is computed using the expansion
\begin{equation}
\Delta s = L\sum_{n=1}^3 \alpha_{c,n} \delta^n + 
\sum_{n=1}^4 R_{5n} x_{\beta,n} +
\sum_{n=1}^2 \left(\frac{\partial^n s}{\partial A_x^n}\right)_0 \frac{A_x^n}{n!} +
\sum_{n=1}^2 \left(\frac{\partial^n s}{\partial A_y^n}\right)_0 \frac{A_y^n}{n!} +
\left(\frac{\partial^2 s}{\partial A_x\partial A_y}\right)_0 A_x A_y
\end{equation}
where $\alpha_{c,1}$ is the linear momentum compaction factor.
Note that in keeping with convention the higher-order momentum compaction is expressed
by polynomial coefficients, not derivatives.
The terms dependent on betatron amplitude are expressed in terms of the more typical
derivatives.
Note the difference between the $R_{5n}$ terms (added in version 2019.4) and those dependent on $A_{x,y}$: the
former are oscillatory while the latter will accumulate. 
The \verb|frequency_map| command can be used to compute path-length dependence on betatron
amplitude.

Using this element is very similar to using the \verb|setup_linear_chromatic_tracking| command.
The advantage is that using {\tt LMATRIX}, one can split a ring into segments
and place, for example, impedance elements between the segments.

This element was inspired by requests from Y. Chae (APS).

N.B.: There is a bug related to using {\tt ILMATRIX} that will result in a crash
if one does not request computation of the twiss parameters. If you encounter this
problem, just add the following statement after the \verb|run_setup| command:
\begin{verbatim}
&twiss_output
        matched = 1
&end
\end{verbatim}

