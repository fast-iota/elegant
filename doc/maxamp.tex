This element sets the aperture for itself and all subsequent elements.
The settings are in force until another {\tt MAXAMP} element is seen.
Settings are also enforced inside of \verb|KQUAD|, \verb|KSEXT|, \verb|KOCT|, \verb|KQUSE|, \verb|CSBEND|, and \verb|CSRCSBEND| elements.

This can introduce unexpected behavior when beamlines are reflected.
For example, consider the beamline
\begin{verbatim}
...
L1:  LINE=( ... )
L2:  LINE=( ... )
MA1: MAXAMP,X_MAX=0.01,Y_MAX=0.005
MA2: MAXAMP,X_MAX=0.01,Y_MAX=0.002
BL1: LINE=(MA1,L1,MA2,L2)
BL:  LINE=(BL1,-BL1)
\end{verbatim}

This is equivalent to
\begin{verbatim}
BL:  LINE=(MA1,L1,MA2,L2,-L2,MA2,-L1,MA1)
\end{verbatim}
Note that the aperture {\tt MA1} is the aperture for all of the first
instance of beamline {\tt L1}, but that {\tt MA2} is the aperture for
the second instance, {\tt -L1}.  This is probably not what was
intended.  To prevent this, it is recommended to always use {\tt
MAXAMP} elements in pairs:
\begin{verbatim}
BL1: LINE=(MA2,MA1,L1,MA1,MA2,L2)
BL:  LINE=(BL1,-BL1)
\end{verbatim}
which is equivalent to
\begin{verbatim}
BL:  LINE=(MA2,MA1,L1,MA1,MA2,L2,-L2,MA2,MA1,-L1,MA1,MA2)
\end{verbatim}
Now, both instances of {\tt L1} have the aperture defined by 
{\tt MA1} and both instances of {\tt L2} have the aperture defined
by {\tt MA2}.

The default values of \verb|X_MAX| and \verb|Y_MAX| are 0, which causes the aperture to be ignored. This means one
cannot use the \verb|MAXAMP| element to simulate a completely blocked beam pipe.
