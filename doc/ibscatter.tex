This element is used for simulating intra-beam scattering
(IBS) effect. The IBS algorithm is based on the Bjorken and
Mtingwa's~\cite{BM} formula, and with an extension of including
vertical dispersion. It can be used for both storage ring and Linac.

To initialize IBS calculation, one or more IBSCATTER elements must be
inserted into the beamline. {\tt elegant} calculates the integrated
IBS growth rates between IBSCATTERs (or from beginning of the beamline
to the first IBSCATTER), then scatter particles at each IBSCATTER
element. Beam's parameters are updated for use in downstream elements.

This method requires that IBSCATTER can not be installed at the beginning
of beamline. The number of other elements between IBSCATTERs or from the beginning of
beamline to the first IBSCATTER has to be 2 or more. For storage ring, an
IBSCATTER must be installed at the end of beamline.

Because the IBS growth rates are energy dependent, special caution is
needed for calculations with accelerating beam. The user needs to
split their accelerating cavity into several pieces, so that $\gamma$
has no large changes between elements.

The user can examine the calculation through an optional SDDS output
file - {\it filename}. The file has a multiple page structure. Each
slice at pass $i$ at each IBSCATTER element occupies one page. Each
page contains integrated IBS growth rates between IBSCATTERs (or from
beginning of the beamline to first IBSCATTER) as parameters, and local
rates for elements in between as tabular data.
