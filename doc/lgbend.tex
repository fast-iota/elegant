This element provides a symplectic straight-pole, multi-segment bending magnet with the exact
Hamiltonian in Cartesian coordinates.
The quadrupole, sextupole, and other multipole terms are defined in Cartesian coordinates.
The element is restricted to having rectangular ends for each segment.
It is, in essence, like a series of \verb|CCBEND| \cite{Borland-LS356} elements concatenated into a whole.

The \verb|LGBEND| element has relatively few explicit parameters, giving the illusion of simplicity.
A custom configuration file is used to specify the many parameters of an \verb|LGBEND|.
This file is generated using the companion program \verb|straightDipoleFringeCalc| from 
a generalized gradient expansion (GGE). The  GGE can be
created using either \verb|computeCBGGE| (for cylindrical-boundary data) or \verb|computeRBGGE|
for (rectangular-boundary data). There is an example in the {\tt elegant} examples collection.

One issue with \verb|LGBEND|, as with \verb|CCBEND| and to a lesser degree \verb|CSBEND|, is that 
the final reference trajectory is not guaranteed to be on axis.
To address this, but default \verb|LGBEND| will automatically adjust the strength of the
first and last segments to minimize the $x$ and $x^\prime$ coordinates of the reference particle.
(This can be defeated by setting \verb|OPTIMIZE_FSE=0|.)
In doing this, the $K_1$ and $K_2$ values of those segments are by default automatically scaled
to ensure that the integrated quadrupole and sextupole are not chagned. 
(This can be defeated by setting \verb|COMPENSATE_KN=0|.)

{\bf Radiation effects}

If \verb|SYNCH_RAD| is non-zero, classical synchrotron radiation is included in tracking.
Incoherent synchrotron radiation, when requested with {\tt ISR=1},
normally uses gaussian distributions for the excitation of the electrons.
(To exclude ISR for single-particle tracking, set \verb|ISR1PART=0|.)
Setting {\tt USE\_RAD\_DIST=1} invokes a more sophisticated algorithm that
uses correct statistics for the photon energy and number distributions.
In addition, if {\tt USE\_RAD\_DIST=1} one may also set {\tt ADD\_OPENING\_ANGLE=1},
which includes the photon angular distribution when computing the effect on 
the emitting electron.  

If \verb|SYNCH_RAD| and \verb|SR_IN_ORDINARY_MATRIX| are non-zero, classical synchrotron radiation will be included in
the ordinary matrix (e.g., for \verb|twiss_output| and \verb|matrix_output|).  Symplecticity is not assured, but the results may
be interesting nonetheless. A more rigorous approach is to use \verb|moments_output|.
\verb|SR_IN_ORDINARY_MATRIX| does not affect tracking.

{\bf Adding errors}

Misalignments are performed in the body-centered frame using Venturini's method \cite{Venturini2021}
based on the values provided for \verb|DX|, \verb|DY|, \verb|DZ|, \verb|ETILT|, \verb|EPITCH|, 
and \verb|EYAW|.
The \verb|FSE| parameter is used to impart a global fractional strength error, which affects not only
the dipole field but also any quadrupole or sextupole terms.
The \verb|TILT| parameter is used to specify the design orientation of the magnet.

{\bf Matrix generation}

{\tt elegant} will use tracking to determine the transport matrix for \verb|LGBEND| elements, which 
is needed for computation of twiss parameters and other operations.
This can require some time, so {\tt elegant} will cache the matrices and re-use them for
identical elements.
Still, there is a performance benefit to be had from using the parallel version, particularly
when assignment of errors prevents sharing of results among many elements.
