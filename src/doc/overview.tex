% How to create this document on UNIX systems:
%   latex elegant
%   dvi2ps elegant | lpr [-P<postscript-printer-name>]
%
\documentclass[11pt]{article}
\usepackage{hyperref}
\usepackage{longtable}
\pagestyle{plain}
%\voffset=-0.75in
\newenvironment{req}{\begin{equation} \rm}{\end{equation}}
\setlength{\topmargin}{0.15 in}
\setlength{\oddsidemargin}{0 in}
\setlength{\evensidemargin}{0 in} % not applicable anyway
\setlength{\textwidth}{6.5 in}
\setlength{\headheight}{-0.5 in} % for 11pt font size
%\setlength{\footheight}{0 in}
\setlength{\textheight}{9 in}
\begin{document}
\author{Michael Borland}
\title{Overview of {\tt elegant} and SDDS}
\maketitle

\section{Purpose of this document}

The purpose of this document is to briefly summarize the capabilities
available of {\tt elegant} and the SDDS toolkit for simulation of
rings, linacs, and beam transport systems..  The description covers
both the serial and parallel versions.  The intended audience is
potential new users, who may want to know in general terms what {\tt
elegant} can do without having to read the manual.  Hence, we will not
show detailed examples.  For that, refer to the
\htmladdnormallink{example
files}{https://ops.aps.anl.gov/cgi-bin/oagLog4.cgi?name=elegantExamples.tar.gz}
available from our web site.

{\tt elegant} is open source, with both \htmladdnormallink{source code
and executables}{https://www.aps.anl.gov/Accelerator-Operations-Physics/Software\#elegant} available.

More detail as well as the most up-to-date information will be found in the 
\htmladdnormallink{manual}{https://ops.aps.anl.gov/manuals/elegant_latest/elegant.html},
as well as the 
\htmladdnormallink{forum}{https://www3.aps.anl.gov/forums/elegant/}.  Users are encouraged
to join and participate in the forum.  At minimum, users should subscribe to the ``Bugs'' topic,
since this is where bug notifications are posted.

The capabilities of {\tt elegant} for accelerator simulation will be broken down as follows:
\begin{enumerate}
\item Simulation capabilties, for example, the ability to simulate random errors.
\item Physics capabilities, for example, the ability to simulate coherent synchrotron radiation effects.
\item Cooperative capabilities, for example, the ability to read data produced by other programs.
\end{enumerate}

\section{Simulation Capabilities}

The simulation capabilities of {\tt elegant} at the highest level, irrespective of any particular physics, are
listed here.  Generally speaking, these capabilities map into the commands that appear in the main input file.
\begin{enumerate}
\item Tracking of rings, linacs, and transport lines
\item Computation of s-dependent and final properties, including
  \begin{itemize}
  \item Beam moments, from tracking or analysis
  \item Twiss parameters
  \item Matrix elements 
  \item Trajectory/orbit
  \item Trajectory/orbit response matrix
  \item Floor coordinates
  \end{itemize}
\item Aperture determination
  \begin{itemize}
  \item Input of s-dependent aperture
  \item Computation of dynamic acceptance
  \item Computation of s-dependent momentum acceptance
  \end{itemize}
\item Optimization of results of computations and tracking
\item Tracking and computations with errors
  \begin{itemize}
  \item Generate or load random error sets
  \item Perform correction of tunes, chromaticities, trajectory/orbit
  \end{itemize}
\item Control lattice parameters
  \begin{itemize}
  \item Scan parameters in multi-dimensional loops
  \item Alter parameters from values in lattice definition
  \item Load SDDS data to overwrite values in lattice definition
  \item Insert elements into lattice
  \item Sub-divide elements
  \end{itemize}
\item Bunch generation, or reading of bunch data from files
\item Multi-stage simulation
  \begin{itemize}
  \item Load particles from previous tracking run
  \item Load lattice parameters from previous optimization run
  \item Load errors and correction settings from previous error run
  \end{itemize}
\item Time-dependent ramping or modulation
\item Determination of frequency maps for rings
\item Change of particle type (default is electrons)
\end{enumerate}

\section{Physics Capabilities}

The physics capabilities of {\tt elegant} are listed here.  Generally
speaking, these capabilities map into elements that can appear in the
lattice file.
\begin{enumerate}
\item Single-particle dynamics
  \begin{itemize}
  \item Magnetic elements using matrix or symplectic tracking methods, e.g.,
    drifts, dipoles, quadrupoles, sextupoles, higher multipoles, wigglers, alpha magnets, solenoids, etc.
  \item Stray fields and steering correctors
  \item Undulator kick maps\cite{Elleaume1992}.
  \item Classical and quantum synchrotron radiation effects in magnets
  \item Accelerating and deflecting standing-wave rf cavities
  \item Traveling wave linear accelerating cavities
  \item Scattering by materials or using user-defined distributions
  \item Time-dependent dipole and multipole kickers
  \item User-designed transformation using external program
  \item User-specified matrix
  \item Apertures and scrapers
  \end{itemize}
\item Collective dynamics
  \begin{itemize}
  \item Short-range wake fields and impedances.
  \item Long-range wakes due to resonant modes.
  \item Coherent synchrotron radiation.
  \item Longitudinal space-charge impedance \cite{Huang2004}.
  \item Transverse space-charge kicks for rings \cite{Xiao2007A}.
  \item Intrabeam scattering \cite{Xiao2008a}.
  \item Touschek scattering \cite{Xiao2010a}.
  \end{itemize}
\item Other
  \begin{itemize}
  \item Beam position monitors
  \item Phase space output and analysis monitors
  \item Digital feedback systems
  \end{itemize}
\end{enumerate}

\section{Cooperative Capabilities}

One of the strengths of {\tt elegant} is its ability to work cooperatively with other programs.
This is done through the use of SDDS (Self-Describing Data Sets\cite{SDDS1,SDDS2}) files.
Such capabilities include
\begin{enumerate}
\item Use of the general-purpose SDDS toolkit for postprocessing and graphics.  Capabilities include
sophisticated plotting, definition of new quantities using formulae, filtering, cross-referencing, sorting,
one- and two-dimensional histograms, frequency analysis, fitting, etc.
You can find out more about SDDS from the \htmladdnormallink{SDDS Info Page}{https://ops.aps.anl.gov/SDDSInfo.shtml}.
\item Reading particle distribution data from other programs, including ASTRA \cite{ASTRA}, IMPACT \cite{IMPACT}, and TRACK \cite{TRACK}.
\item Phase space analysis to obtain twiss parameters and moments.
\item Computation of radiation brightness and flux tuning curves, as well as radiation distributions and other properties.
\item Upsampling of particle distributions to increase particle number, smooth the distribution, and add modulations.
\item Processing of quadrupole-scan emittance measurement data.
\item Computation of the CSR impedance for use in tracking \cite{Warnock}.
\item Use of {\tt elegant} data for Touschek lifetime, intrabeam scattering, and potential well distortion computations.
\item Translation of {\tt elegant} lattice into other formats.
\item Computation of multipole error data to reflect defined magnet construction errors \cite{Halbach_69a}.
\end{enumerate}


\begin{thebibliography}{9}

\bibitem{SDDS1}
M. Borland, ``A Self-Describing File Protocol for Simulation
Integration and Shared Postprocessors,'' Proc. 1995 PAC, May 1-5,
1995, Dallas, Texas, pp. 2184-2186 (1996).

\bibitem{SDDS2}
M. Borland, ``A Universal Postprocessing Toolkit for Accelerator 
Simulation and Data Analysis,'' Proc. 1998 ICAP Conference, 
Sept. 14-18, 1998, Monterey, California, to be published.

\bibitem{Halbach_69a}
    K. Halbach, ``First Order Perturbation Effects in Iron-Dominated Two-Dimensional Symmetrial Multipoles'',
    NIM {\bf 74-1}, 1969, 147-164.

\bibitem{Huang2004}
  Z. Huang {\em et al.}, Phys. Rev. ST Accel. Beams {\bf 7} 074401 (2004).

\bibitem{Xiao2007A}
  A. Xiao {\em et al.}, ``Direct Space-Charge Calculation in {\tt elegant} and its Application to the
 ILC Damping Ring,'' Proc. PAC2007, 3456-3458.

\bibitem{Xiao2008a}
  A. Xiao, ``Study of IBS Effects for High-Brightness Linac Beams,'' Proc. Linac08, 296-298 (2009).

\bibitem{Xiao2010a}
  A. Xiao {\em et al.}, ``Monte Carlo simulation of Touschek effect,'' Phys. Rev. ST Accel. Beams {\bf 13},
  074201 (2010).

\bibitem{Warnock}
  W. Warnock, ``Shielded Coherent Synchrotron Radiation and Its Effect on Very Short Bunches,'' SLAC-PUB-5375, 1990.

\bibitem{Elleaume1992}
  P. Elleaume, ``A New Approach to Electron Beam Dynamics in Undulators and Wigglers,'' Proc. EPAC 1992, 661-663.

\bibitem{ASTRA}
  K. Fl\"{o}ttmann, Astra User Manual, http://www.desy.de/~mpyflo/Astra\_dokumentation/

\bibitem{IMPACT}
J. Qiang {\em et al.},  J.~Comp.~Phys. 163, 434 (2000).

\bibitem{TRACK}
  V. N. Aseev {\em et al.}, Proc. PAC05, 2053-2055 (2005); ASCII version 39 from B. Mustapha.

\end{thebibliography}

\end{document}
