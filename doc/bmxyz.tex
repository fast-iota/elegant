This element simulates transport through a 3D magnetic field
specified as a field map.  It does this by simply integrating the
Lorentz force equation in cartesian coordinates.  It does not
incorporate changes in the design trajectory resulting from the
fields.  I.e., if you input a dipole field, it is interpreted as a
steering element.

The field map file is an SDDS file with the following columns:
\begin{itemize}
\item {\bf x}, {\bf y}, {\bf x} --- Transverse coordinates in meters (units should be ``m'').
\item {\bf Fx}, {\bf Fy}, {\bf Fx} --- Normalized field values (no units).  The
        field is multiplied by the value of the STRENGTH parameter to convert it to a 
        local bending radius.  For example, an ideal quadrupole could be simulated
        by setting (Fx=y, Fy=x, Fz=0), in which case STRENGTH is the
        K1 quadrupole parameter.
\item {\bf Bx}, {\bf By}, {\bf Bz} --- Field values in Tesla (units should be ``T'').
        The field is still multiplied by the value of the STRENGTH parameter, which
        is dimensionless.
\end{itemize}

The field map file must contain a rectangular grid of points,
equispaced (separately) in x, y, and z.  There should be no missing values
in the grid (this is not checked by {\tt elegant}).  In addition, the
x values must vary fastest as the values are accessed in row order, then the y values.
To ensure that this is the case, use the following command on the field
file:
\begin{flushleft}
sddssort {\em fieldFile} -column=z,incr -column=y,incr -column=x,incr
\end{flushleft}

This element is an alternative to \verb|FTABLE| using a more conventional integration method.

The \verb|BXFACTOR|, \verb|BYFACTOR|, and \verb|BZFACTOR| allow multiplying the indicated field components by the given factors.
These scaling parameters may result in unphysical fields.

By default, the \verb|BMXYZ| element should be supplied with the full 3D field map of the magnet.
To allow saving memory and reducing the file to load data, partial magnetic field maps can be 
loaded as well, but the user must specify the symmetry of the magnet to ensure that the fields are
modeled correctly in the full volume.
This is done using three optional parameters in the input file, \verb|xSymmetry|, \verb|ySymmetry|, and \verb|zSymmetry|.
If present, these must have one of three \verb|none| (default), \verb|even|, and \verb|odd|.  

For example, a normal quadrupole magnet would have
\verb|xSymmetry=odd|, \verb|ySymmetry=odd|, and \verb|zSymmetry=even|.  A left/right symmetric dipole or sextupole
would have \verb|xSymmetry=even|, \verb|ySymmetry=odd|,
\verb|zSymmetry=even|.  A normal octupole would have the same symmetry codes as a normal quadrupole.
Note that when using these symmetries, the user is not required to limit the field map to, say, 
$x\geq 0$, $y\geq 0$, and $z\geq 0$, though doing so saves the most memory.
If possible, it is recommended to provide the fields over $x\geq -2\Delta x$, $y\geq -2\Delta y$, 
and $z\geq 0$, so that the transverse interpolation as a few points on both sides of the origin.
This will ensure better results near the origin.

Internally, {\tt elegant} stores only the partial field map. When field values are required outside this region, the
declared symmetries are used to map the coordinates to the covered region and change signs of the various field
components if required.

Internal apertures may be specified using four methods, as used in elements like \verb|KQUAD| and
\verb|CSBEND|. The methods include upstrea \verb|MAXAMP| elements, upstream \verb|APCONTOUR| elements
with \verb|STICKY=1|, $s$-dependent apertures defined via the \verb|aperture_data| command,
and global-coordinate system apertures defined via the \verb|obstruction_data| command.

For \verb|APCONTOUR|-defined apertures, this can be invoked using the \verb|STICKY=1| parameter,
which is used to impose the aperture contour on downstream elements.
One fine point is that the field map is typically significantly longer than the magnet itself.
We don't want to apply the apertures except in some region in the interior of the field map
This can be specified using the \verb|ZMIN_APCONTOUR| and \verb|ZMAX_APCONTOUR| parameters,
which give the range of application in the coordinate system of the field map.
In this case, to prevent application of the \verb|APCONTOUR| apertures at the point of 
definition, the \verb|APCONTOUR| element should have \verb|HOLDOFF=1| and be inserted
in the lattice just before the \verb|BMXYZ| element.

