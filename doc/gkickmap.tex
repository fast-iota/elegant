This element provides simulation a generalized kick map, similar to 
the \verb|UKICKMAP| but appropriate for maps that do not pertain to undulators or
wigglers.

The input file has the following columns:
\begin{itemize}
\item \verb|x| --- Horizontal position in meters.
\item \verb|y| --- Vertical position in meters.
\item \verb|xpFactor| --- Dimensionless horizontal kick factor.  The horizontal kick for any particle with
  a particular momentum deviation $\delta$ is the interpolated value of \verb|xpFactor| divided by $1+\delta$.
\item \verb|ypFactor| --- Dimensionless horizontal kick factor. The vertical kick for any particle with
  a particular momentum deviation $\delta$ is the interpolated value of \verb|ypFactor| divided by $1+\delta$.
\end{itemize}
The values of \verb|x| and \verb|y| must be laid out on a grid of equispaced points.
It is assumed that the data is ordered such that \verb|x| varies fastest.  This can be
accomplished with the command
\begin{verbatim}
% sddssort -column=y,increasing -column=x,increasing input1.sdds input2.sdds
\end{verbatim}
where \verb|input1.sdds| is the original (unordered) file and \verb|input2.sdds| is the
new file, which would be used with \verb|KICKMAP|.

This element is included in beam moments computations via the \verb|moments_output| command.

The \verb|YAW| and \verb|YAW_END| parameters can be used in the simulation of canted IDs.
Normally, steering magnets are used to create an angle between the devices.
The devices are thus oriented in the reference coordinate system, meaning the beam tranverses
the IDs at an angle.
If it is desirable to align the IDs to the beam, the IDs can be yawed. A positive yaw will
tilt the ID so that it is colinear with a beam that has been kicked by a positive horizontal
steering angle.
The \verb|YAW_END| parameter defines which end of the ID is held fixed when the yaw is applied.

