This element simulates transport through a transverse magnetic field
specified as a field map.  It does this by simply integrating the
Lorentz force equation in cartesian coordinates.  It does not
incorporate changes in the design trajectory resulting from the
fields.  I.e., if you input a dipole field, it is interpreted as a
steering element.

The field map file is an SDDS file with the following columns:
\begin{itemize}
\item {\bf x}, {\bf y} --- Transverse coordinates in meters (units should be ``m'').
\item {\bf Fx}, {\bf Fy} --- Normalized field values (no units).  The
        field is multiplied by the value of the STRENGTH parameter to convert it to a 
        local bending radius.  For example, if Fx=y and Fy=x, then STRENGTH is the
        K1 quadrupole parameter.
\item {\bf Bx}, {\bf By} --- Field values in Tesla (units should be ``T'').
        The field is still multiplied by the value of the STRENGTH parameter, which
        is dimensionless.  Note: the default value of STRENGTH is 0, so if you don't
        set it to something, you'll get no effect!
\end{itemize}

The field map file must contain a rectangular grid of points,
equispaced (separately) in x and y.  There should be no missing values
in the grid (this is not checked by {\tt elegant}).  In addition, the
x values must vary fastest as the values are accessed in row order.
To ensure that this is the case, use the following command on the field
file:
\begin{flushleft}
sddssort {\em fieldFile} -column=y,incr -column=x,incr
\end{flushleft}

