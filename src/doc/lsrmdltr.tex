This element simulates a planar undulator, together with an optional
co-propagating laser beam that can be used as a beam heater or
modulator.  The simulation is done by numerical integration of the
Lorentz equation.  It is not symplectic, and hence this element is not
recommended for long-term tracking simulation of undulators in storage
rings.  

The fields in the undulator can be expressed in one of three ways.
The FIELD\_EXPANSION parameter is used to control which method is used.
\begin{itemize}
\item The exact field, given by (see section 3.1.5 of the {\em Handbook of
Accelerator Physics and Engineering})
\begin{equation}
B_x = 0,
\end{equation}
\begin{equation}
B_y = B_0 \cosh k_u y \cos k_u z,
\end{equation}
and
\begin{equation}
B_z = -B_0 \sinh k_u y \sin k_u z ,
\end{equation}
where $k_u = 2\pi/\lambda_u$ and $\lambda_u$ is the undulator period.
This is the most precise method, but also the slowest.  

{\bf Experimental feature:} One may also model a transverse gradient undulator (TGU) by setting the \verb|TGU_GRADIENT| parameter
to a non-zero value.
In this case, taking $a$ as the normalized gradient, the fields are \cite{RLindbergPC}
\begin{equation}
B_x = \frac{a B_0 \sinh k_u y \cos k_u z}{k_u},
\end{equation}
\begin{equation}
B_y = B_0 \left((1 + a x) \cosh k_u y \cos k_u z + \frac{a C }{2 k_u^2}\frac{e B_0}{\gamma m_e c }\right)
\end{equation}
and
\begin{equation}
B_z = -B_0 (1 + a x) \sinh k_u y \sin k_u z,
\end{equation}
where $\gamma$ is the central relativistic factor for the beam and $C$ is given by the
\verb|TGU_COMP_FACTOR| parameter.
This factor, and the term it multiplies, is present in order to help suppress the 
trajectory error at the end of the device.
It may require adjustment in order to achieve the desired level of correction.
In addition, the user may need to adjust the pole-strength factors and include external misalignments
and steering magnets in order to supress not only the trajectory error, but also dispersion errors.

\item The field expanded to leading order in $y$:
\begin{equation}
B_y = B_0 ( 1 + \frac{1}{2}(k_u y)^2 ) \cos k_u z,
\end{equation}
and
\begin{equation}
B_z = -B_0 k_u y \sin k_u z.
\end{equation}
In most cases, this gives results that are very close to the exact fields,
at a savings of 10\% in computation time.
This is the default mode.

\item The ``ideal'' field:
\begin{equation}
B_y = B_0 \cos k_u z,
\end{equation}
\begin{equation}
B_z = -B_0 k_u y \sin k_u z.
\end{equation}
This is about 10\% faster than the leading-order mode, but less
precise.  Also, {\em it does not include vertical focusing}, so it is
not generally recommended.
\end{itemize}

If \verb|HELICAL| is set to a nonzero value, a helical device is modeled by combining the
fields of two planar devices, one of which is rotated 90 degrees and displaced one quarter
wavelength.
Again, the FIELD\_EXPANSION parameter is used to control which method is used.
\begin{itemize}
\item The exact fields are
\begin{equation}
B_x = -B_0 \cosh k_u x \sin k_u z ,
\end{equation}
\begin{equation}
B_y = B_0 \cosh k_u y \cos k_u z,
\end{equation}
and
\begin{equation}
B_z = -B_0 \sinh k_u y \sin k_u z -B_0 \sinh k_u x \cos k_u z,
\end{equation}

\item The field expanded to leading order in $x$ and $y$:
\begin{equation}
B_x = -B_0 ( 1 + \frac{1}{2}(k_u x)^2 ) \sin k_u z,
\end{equation}
\begin{equation}
B_y = B_0 ( 1 + \frac{1}{2}(k_u y)^2 ) \cos k_u z,
\end{equation}
and
\begin{equation}
B_z = -B_0 k_u y \sin k_u z -B_0 k_u x \cos k_u z.
\end{equation}

\item The ``ideal'' field is
\begin{equation}
B_x = -B_0 \sin k_u z,
\end{equation}
\begin{equation}
B_y = B_0 \cos k_u z,
\end{equation}
\begin{equation}
B_z = 0
\end{equation}
This is about 10\% faster than the leading-order mode, but less
precise.  Also, {\em it does not include vertical focusing}, so it is
not generally recommended.

\end{itemize}

The expressions for the laser field used by this element are from
A. Chao's article ``Laser Acceleration --- Focussed Laser,'' available
on-line at \\
http://www.slac.stanford.edu/$\sim$achao/LaserAccelerationFocussed.pdf .
The implementation covers laser modes TEM{\em ij}, where
$0\leq i \leq 4$ and $0 \leq j \leq 4$.

By default, if the laser wavelength is not given, it is computed from the resonance
condition:
\begin{equation}
\lambda_l = \frac{\lambda_u}{2 \gamma^2} \left( 1 + \frac{1}{2} K^2 \right),
\end{equation}
where $\gamma$ is the relativistic factor for the beam and $K$ is the
undulator parameter.

The adaptive integrator doesn't work well for this element, probably
due to sudden changes in field derivatives in the first and last three
poles (a result of the implementation of the undulator terminations).
Hence, the default integrator is non-adaptive Runge-Kutta.  The
integration accuracy is controlled via the N\_STEPS parameter.
N\_STEPS should be about 100 times the number of undulator periods.

The three pole factors are defined so that the trajectory is centered
about $x=0$ and $x^\prime=0$ with zero dispersion.  This wouldn't be
true with the standard two-pole termination, which might cause problems 
overlapping the laser with the electron beam.

The laser time profile can be specified using the \verb|TIME_PROFILE|
parameter to specify the name of an SDDS file containing the
profile.   If given, the electric and magnetic fields of the laser are
multiplied by the profile $P(t)$.  Hence, the laser intensity is multiplied
by $P^2(t)$. By default $t=0$ in the
profile is lined up with $\langle t \rangle$ in the electron bunch.
This can be changed with the \verb|TIME_OFFSET| parameter. A positive
value of \verb|TIME_OFFSET| moves the laser profile forward in time (toward
the head of the bunch).
