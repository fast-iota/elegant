\documentclass[11pt]{article}
\usepackage{html}
\pagestyle{plain}
%\voffset=-0.75in
\newenvironment{req}{\begin{equation} \rm}{\end{equation}}
\setlength{\topmargin}{0.15 in}
\setlength{\oddsidemargin}{0 in}
\setlength{\evensidemargin}{0 in} % not applicable anyway
\setlength{\textwidth}{6.5 in}
\setlength{\headheight}{-0.5 in} % for 11pt font size
%\setlength{\footheight}{0 in}
\setlength{\textheight}{9 in}
\begin{document}

\title{Change Notes for Version 14.4 of elegant}
\author{Advanced Photon Source\\Michael Borland\\ \date{\today}}
\maketitle

Version 14.4 of elegant is now available.  The most significant
enhancement is to the CSRDRIFT element, which now provides more
rigorous simulation of CSR in drift spaces.  There is also support for
body focusing in RFCA and RFCW elements, and two new elements for
simulating RF cavities and solenoids from field maps.

\begin{enumerate}
\item The CSRDRIFT element has a number of new features.  In order of
increasing significance, these are:
\begin{enumerate}
\item  Added the OL\_MULTIPLIER parameter, which defaults to 1.  If you
give \\ USE\_OVERTAKING\_LENGTH=1, then elegant uses the overtaking length
as the attenuation length for the CSR wake.  This parameter permits
you to increase or decrease that attenuation length by the specified
factor.

\item  Added the BUNCHLENGTH\_MODE parameter, so that the user can choose
how the bunch length is computed for computation of the attenuation
length.  Nominally, one should use the true RMS, but when the beam has
temporal spikes, it isn't always clear that this is the best choice.
The choices are ``rms'', ``68-percentile'', and ``90-percentile''.  The last
two imply using half the length determined from the given percentile
in place of the rms bunch length.  I usually use 68-percentile, which
is the default.
\item  Added the USE\_SALDIN54 parameter, which invokes use of a more
rigorous method of computing the fall-off of the CSR wake in the
drift.  As the name suggests, it is based on equation 54 (and 53) from
Saldin et al. (NIM A 398 (1997)).  I no longer use any other mode for
this element.  Using USE\_OVERTAKING\_LENGTH tends to give similar
results, however.  I also added the SALDIN54\_OUTPUT parameter, which
allows one to specify a file into which the wake intensity vs z will
be written.  These equations are for a rectangular bunch; the length 
of this rectangular bunch is taken to be twice the bunch length
computed according to the BUNCHLENGTH\_MODE parameter.  If your 
bunch is nearly rectangular, then you probably want BUNCHLENGTH\_MODE
of ``90-percentile''.
\end{enumerate}

\item It's now possible to assign multiple errors to one item and get a
"final" output file at the same time.  In previous versions, this
would cause the program to terminate.

\item Added the "string\_value" parameter to alter\_elements command,
allowing alteration of string parameters of elements.

\item A bug was introduced in version 14.3 for the beam generation
with bunched\_beam.  In particular, pair-wise randomization of the
generated coordinates was performed by default.  This has been fixed.
The randomize\_order parameter of the bunched\_beam namelist takes
three possible values: 0 means no randomization; 1 means randomize (x,
x', y, y', t, p) values independently, which destroys any x-x', y-y',
and t-p correlations; 2 means randomize (x, x'), (y, y'), and (t, p)
in pair-wise fashion, which retains correlations between x-x' etc but
destroys those between x-y, x-p, etc.  This is used with Halton
sequences to remove banding.  Be aware that if you provide values for
the dispersion or its slope in the beam definition, that these will be
nulled out by use of either type of randomization.

\item The bunched\_beam command now allows specifying the longitudinal
Twiss parameters, emit\_z, beta\_z, and alpha\_z, instead of the
energy spread, bunch length, and normalized correlation.  You may also
specify the energy spread, bunch length, and alpha\_z together,
avoiding the use of the troublesome dp\_s\_coupling parameter.
Basically, which values {\tt elegant} uses depends what you set to
nonzero values.  If you set emit\_z, then sigma\_dp, sigma\_s, and
dp\_s\_coupling are ignored.  If you don't set emit\_z, then {\tt
elegant} uses sigma\_dp and sigma\_s; it additionally uses alpha\_z if
it is nonzero, otherwise it uses dp\_s\_coupling.  Paul Emma pointed out
that using alpha\_z is easier than dp\_s\_coupling some time ago and I
just got around to adding it.  For reference, the relationship between
them is $ C = \frac{\Sigma_{56}}{\sqrt{\Sigma_{55}\Sigma_{66}}} =
-\frac{\alpha}{\sqrt{1+\alpha^2}}$.  Note that to impart a
chirp that results in compression for $R_{56}<0$, you must have
$\alpha_z<0$ or $C>0$.

\item The MAPSOLENOID element was added.  It provides numerical integration 
through solenoid fields specified as Br(z,r) and Bz(z,r).  Parameters
are as follows:
\begin{enumerate}
\item L               length in meters
\item DX, DY          misalignment in meters
\item N\_STEPS         number of steps for nonadaptive integration
\item INPUTFILE       name of SDDS file containing (Br, Bz) vs (r, z).  
                Each page should have values for a fixed r.
\item RCOLUMN         name of column containing r values
\item ZCOLUMN         name of column containing z values
\item BRCOLUMN        name of column containing Br values
\item BZCOLUMN        name of column containing Bz values
\item FACTOR          factor by which to multiply fields to get applied fields in Tesla
\item ACCURACY        integration accuracy for adaptive integration (not recommended)
\item METHOD          integration method (runge-kutta, bulirsch-stoer, non-adaptive 
                runge-kutta, modified midpoint).  Recommend non-adaptive runge-kutta.
\end{enumerate}

\item The RFTMEZ0 element was added.  This element provides numerical integration
through a TM-mode RF cavity, where the fields are specified by Ez(r=0) and an
off-axis expansion.  Parameters are as follows:
\begin{enumerate}
\item L               length in meters
\item FREQUENCY       frequency in Hz
\item PHASE           in radians
\item EZ\_PEAK         Peak on-axis longitudinal electric field at crest (V/m)
\item TIME\_OFFSET     time offset (adds to phase)
\item PHASE\_REFERENCE phase reference number (to link to other time-dependent elements)
\item DX, DY          misalignment in meters
\item N\_STEPS         number of steps for nonadaptive integration
\item RADIAL\_ORDER    highest order in off-axis expansion
\item CHANGE\_P0       does element change central momentum?
\item INPUTFILE       name of SDDS file containing Ez vs z at r=0
\item ZCOLUMN         name of column containing z values
\item EZCOLUMN        name of column containing Ez values
\item SOLENOID\_FILE   Name of SDDS file containing map of Bz and Br vs z and r.  
                Each page contains values for a single r.  This is the same
                organization as the file used by MAPSOLENOID.
\item SOLENOID\_ZCOLUMN, SOLENOID\_RCOLUMN, SOLENOID\_BZCOLUMN, \\
SOLENOID\_BRCOLUMN, SOLENOID\_FACTOR
                These all correspond to the similarly-named parameters of
                MAPSOLENOID.
\item ACCURACY        integration accuracy for adaptive integration
\item METHOD          integration method (runge-kutta, bulirsch-stoer, non-adaptive 
                runge-kutta, modified midpoint).  Recommend non-adaptive runge-kutta.
\item FIDUCIAL  Specifies how to determine the fidicual particle.  Syntax is
                \verb|<coord>,<method>|, where coord is either 't' or 'p'
                and method is one of median, min, max, ave, first, or light.  E.g., 
                \verb|t,median| means
                that the fiducial particle has the same arrival time as the
                       median arrival time of the bunch.
\end{enumerate}

The RFTMEZ0 and MAPSOLENOID elements were requested by Eliane Lessner, who
helped in the testing.

\item  Added the BEAMCENTERED parameter (flag) to the PFILTER element. If set to
a nonzero value, then momentum filtering is done relative to the average beam
momentum, rather than relative to the nominal central momentum of the beamline.

\item Added the CSR parameter (flag) to CSRCSBEND.  If set to 0, then CSR is turned
off in that element.

\item Fixed a bug in computation of the Twiss phase advance for elements
that have negative drift length.  This bug was discovered by Paul Emma.

\item The RFCA and RFCW elements now support rf focusing in the body
of the element (i.e., not end focusing) using a simplification of
Rosenzweig and Serafini's results (Phys. Rev. E 49 (2), 1599).  The
simplification is to assume that the field is due to a "pure pi-mode
standing-wave".  This was suggested by N. Towne (NSLS), and I used his
results in checking the new feature.  To activate this feature, set
the BODY\_FOCUS\_MODEL parameter to "SRS" (Simplfied Rosenzweig
Serafini).  The default value of this parameter is "none."  I'm
willing to add other models using the Rosenzweig/Serafini formulation
if people supply me with the parameters.  Please note that this
feature is independent of the existing end-focusing control through
the END1\_FOCUS and END2\_FOCUS flags.  This addition was suggested by
Nathan Towne, and I used his note on the subject to help debug the new
code.

\item A bug was fixed in the link\_elements feature that resulted in
double-counting of incremental changes implemented via links.  As the
manual states, the prior value of any linked-to quantity is on the
stack when the equation is executed.  Unforunately, links must be
executed twice, so if you used that prior value you'd end up with
unexpected results.  This bug was discovered by Mark Woodley.

\item A bug was fixed in the load\_parameters feature that resulted in
failure to load the parameter values on the last step of a run with
errors.

\item {\tt elegant} now supports macro substitution in the input file.
This permits making runs with altered parameters without editing the input
file.  Macros inside the input file have one of two forms: \verb|<tag>|
or \verb|$tag|.  To perform substitution, use the syntax
\begin{flushleft}{\tt
elegant {\em inputfile} -macro={\em tag1}={\em value1}[,{\em tag2}={\em value2}...]
}\end{flushleft}
When using this feature, it is important to substitute the rootname
(in run\_setup) so that you can get a new set of output files (assuming
you use the suggested ``\%s'' field in all the output file names).

\item A number of changes were made to the program sddsmatchtwiss: The
-xPlane and -yPlane options now permit changing the emittances of the
beam.  I also added the -zPlane option, which permits matching the
parameters of the longitudinal phase space. The new -oneTransform
option allows computing the transform only for the first page, and
using it for all pages; this is useful if you want to simulate
correction of the beam properties with jitter about the correction.

\end{enumerate}
\end{document}

