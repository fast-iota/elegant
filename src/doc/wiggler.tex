This element simulates a wiggler or undulator.  There are two aspects
to the simulation: the effect on radiation integrals and the vertical
focusing.  Both are included as of release 15.2 of elegant.  

If the number of poles should be an odd integer, we include
half-strength end poles to match the dispersion, but only for the
radiation integral calculation.  For the focusing, we assume all the
poles are full strength (i.e., a pure sinusoidal variation).  If the
number of poles is an even integer, no special end poles are required,
but we make the unphysical assumption that the field at the entrance
(exit) of the device jumps instantaneously from 0 (full field) to full
field (0).

The radiation integrals were computed analytically using Mathematica,
including the variation of the horizontal beta function and dispersion. 
For an odd number of poles, half-strength end-poles are assumed in order
to match the dispersion of the wiggler.  For an even number of poles,
half-length end poles are assumed (i.e., we start and end in 
the middle of a pole), for the same reason.

The vertical focusing is implemented as a distributed quadrupole-like
term (affecting ony the vertical, unlike a true quadrupole).  The
strength of the quadrupole is (see Wiedemann, {\em Particle Accelerator
Physics II}, section 2.3.2)
\begin{equation}
K_1 = \frac{1}{2\rho^2},
\end{equation}
where $\rho$ is the bending radius at the center of a pole.  The
undulator is focusing in the vertical plane.

The wiggler field strength may be specified either as a peak bending 
radius $\rho$ (RADIUS parameter) or using the dimensionless strength parameter
K (K parameter).  These are related by
\begin{equation}
K = \frac{\gamma \lambda_u}{2 \pi \rho},
\end{equation}
where $\gamma$ is the relativistic factor for the beam and $\lambda_u$ is
the period length.

